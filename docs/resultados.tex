\section{Resultados}	
A continuaci\'on presentaremos los distintos gr\'aficos y analisis realizados, en relacion a las 4 universidades elegidas. Merece nombrarse, que se habia elegido una universidad m\'as, llamada University of Pretoria.

En cuanto a esta ultima universidad nombrada, nos encontramos con que no pudimos rastrear su ruta. Lo que obtuvimos en nuestros intentos, fue siempre el mismo resultado. Incrementamos el $ttl$ hasta obtener un $ubyte$ overflow. El valor de nuestro $ttl$ se fue del rango $0 < ttl < 255$.
Por este motivo no utilizamos este caso de estudio en los gr\'aficos y analisis.


\subsection{Primera parte: selección de las universidades}
Las 4 universidades de distintas partes del mundo seleccionadas para llevar a cabo el análisis requerido son las siguientes: 

\begin{itemize}
 \item {\bf The University British of Columbia}
 
	{\bf Distancia}: 11302.14 km 
	
	{\bf IP}: 137.82.130.49 (\url{www.ubc.ca}{})
 
 \item {\bf Lomonosov Moscow State University}

	{\bf Distancia}: 13481.01 km
	
	{\bf IP}: 93.180.0.18 (\url{www.msu.ru}{})
 
 \item {\bf The Chinese University of Hong Kong}

	{\bf Distancia}: 18511.04 km
	
	{\bf IP}: 137.189.11.73 (\url{www.cuhk.edu.hk})
 
 \item {\bf National University Of Samoa}
 
	{\bf Distancia}: 10771.06 km
	
	{\bf IP}: 23.229.137.67 (\url{www.nus.edu.ws/}{})
 
\end{itemize}

Las distancias descriptas corresponden a la distancia lineal que hay entre la universidad y un punto en común en Buenos Aires.


\subsection{Segunda Parte: Correr nuestro Traceroute}

Se ejecutó el traceroute implementado por nosotros utilizando Scapy a lo largo de un día, con intervalos de una hora. El traceroute envía 20 paquetes a cada hop, para lograr un promedio más ajustado del RTT.

Evaluamos distintas API's para calcular las coordenadas y/o la localización de las IP's con el objetivo de que nuestros datos sean lo más precisos posibles.


Los resultados obtenidos se muestran en tres gráficos distintos por cada universidad, cada uno representando una corrida diferente con una API distinta. En cada gráfico hay una línea por cada hora en la que se ejecutó el traceroute marcando los distintos valores de los RTT's en función de los TTL's. A continuación se muestran los gráficos:

%%%%%%%%%%%%%%%%%%%%%%%%%%%%%%% GRAFICOS %%%%%%%%%%%%%%%%%%%%%%%%%%%%%%%%%%%%%%%%
\ponerGrafico{./graficos/canada_dani.png}{Canada Muestra 1}{0.4}{canada_dani}

\ponerGrafico{./graficos/canada_santi.png}{Canada Muestra 2}{0.4}{canada_santi}

\ponerGrafico{./graficos/canada_juan.png}{Canada Muestra 3}{0.4}{canada_juan}
 

\ponerGrafico{./graficos/china_dani.png}{Canada Muestra 1}{0.4}{china_dani}
\ponerGrafico{./graficos/china_santi.png}{Canada Muestra 2}{0.4}{china_santi}
\ponerGrafico{./graficos/china_juan.png}{Canada Muestra 3}{0.4}{china_juan}

 
\ponerGrafico{./graficos/rusia_dani.png}{Canada Muestra 1}{0.4}{rusia_dani}
\ponerGrafico{./graficos/rusia_santi.png}{Canada Muestra 2}{0.4}{rusia_santi}
\ponerGrafico{./graficos/rusia_juan.png}{Canada Muestra 3}{0.4}{rusia_juan}

 
\ponerGrafico{./graficos/samoa_dani.png}{Canada Muestra 1}{0.4}{samoa_dani}
\ponerGrafico{./graficos/samoa_santi.png}{Canada Muestra 2}{0.4}{samoa_santi}
\ponerGrafico{./graficos/samoa_juan.png}{Canada Muestra 3}{0.4}{samoa_juan}

Para todas las API's se puede observar un aumento considerable del RTT entre dos TLL's específicas (distintas según cada gráfico) que representa una transmisión de gran escala. 

Por ejemplo, en el gráfico \ref{fig:canada_santi}, el RTT tiene un aumento vertiginoso entre el TLL 4 y el 5, que es cuando los paquetes llegan a Estados Unidos. Sucede algo similar en \ref{fig:canada_dani} sólo que se da entre el TLL 7 y el 8 en algunos casos y entre el 8 y el 9 en otros. También en ambos casos según cada corrida del traceroute coincide con el momento en que se envían paquetes a Estados Unidos. En \ref{fig:canada_juan} se da una situación similar entre los TLL's 6 y 7 y 8 y 9 también coincidiendo con un salto de los paquetes entre Argentina y Estados Unidos

Con respecto a las pruebas realizadas para China, en el gráfico \ref{fig:china_santi} se observan dos saltos claros entre los TLL's 4 y 5 y 6 y 7 que coinciden con dos saltos entre distintas regiones de Estados Unidos. Curiosamente, el salto entre Estados Unidos y Hong Kong no presenta un RTT mucho mayor que aquel que iba a dos regiones distintas de Estados Unidos. Se puede observar algo similar en el gráfico \ref{fig:china_dani }y \ref{fig:china_juan} en el mismo salto (que tiene como destino una IP perteneciente a la ciudad de Miami) sólo que en estos casos es entre las TLL's 9 y 10.

Además, varios gráficos presentan picos de RTT injustificados para una hora determinada. Este puede ser el caso de los gráficos \ref{fig:rusia_juan} y \ref{fig:canada_juan}, y consideramos estos picos como producto de alguna anomalía, dado que sólo ocurrieron una única vez en todas las pruebas. Por otro lado, los paquetes que no obtuvieron respuesta se pueden apreciar como un pico decreciente que llega hasta el valor 0 (que es como se representa a los paquetes que nunca retornaron).


\subsection{Segunda parte: búsqueda de enlaces transatlánticos}

Es esperable que los enlaces transatl\'anticos tengan la mayor diferencia de rtt entre salto y salto, ya que son lo que probablemente tengan una mayor distancia entre ellos. Asique en esta secci\'on vamos a tratar de ubicar a estos hops distinguidos. El enunciado nos propone que utilicemos el z-score para tratar de encontrarlos. 

El $ZRTT$ se define de la siguiente manera: 

\begin{equation}
 ZRTT_i = \frac{RTT_i - \overline{RTT}}{SRTT}
\end{equation}

siendo $\overline{RTT}$ y $SRTT$ el promedio y el desv\'io standard de los RTTs de la ruta, respectivamente, y $RTT_i$ al RTT medido para el salto $i$.

Luego nos piden encontrar un umbral, que sea una cota inferior para los zrtt de los enlaces transatl\'anticos. 

A continuaci\'on mostraremos los gr\'aficos obtenidos para las universidades, a partir de una experimentaci\'on en la que calculamos el rtt y zrtt y los comparamos en un mismo gr\'afico: 

\ponerGrafico{./graficos/canada_rtt_dani.png}{Canada}{0.5}{canada_zrtt}
\ponerGrafico{./graficos/china_rtt_04_dani.png}{China}{0.5}{china_zrtt}
\ponerGrafico{./graficos/rusia_rtt_07_santi.png}{Rusia}{0.5}{rusia_zrtt}
\ponerGrafico{./graficos/samoa_rtt_07_santi.png}{Samoa}{0.5}{samoa_zrtt}

De los mismos, podemos ver que el zrtt funciona como una normalizaci\'on de los rtt tomados, y que en todos se cumple que los saltos transatl\'anticos tienen un z-score mayor a $0.5$, siendo este el umbral que nos piden. 

\subsection{Mapas}
\ponerGrafico{./mapas/Canada-Dani.png}{Canada}{0.5}{CanadaIp}
\ponerGrafico{./mapas/Hong-Kong.png}{China}{0.4}{}
\ponerGrafico{./mapas/Rusia.png}{Rusia}{0.5}{}
\ponerGrafico{./mapas/Samoa.png}{Samoa}{0.6}{}

