Este trabajo pr\'actico fue realizado con el objetivo de entender el protocolo ICMP y de implementar nuestra propia herramienta de traceroute utilizando este protocolo de control, para finalmente poder realizar un analisis sobre los resultados obtenidos que nos permita entender un poco mejor que es lo que realmente sucede.
Con ese fin se utilizan diversas herramientas y técnicas. Una de ellas consiste en el $traceroute$ brindado por el sistema operativo y otra la desarrolla en Scapy. 

En primer lugar se seleccionaron tres universidades situadas en otros diferentes continentes. Una de ellas 
es The University of British Columbia, ubicada en la ciudad de Vancouver en Canadá. La segunda es The University of Hong Kong, ubicada en Pokfulam, Hong Kong. Otra de las universidades seleccionadas es Lomonosov Moscow State University, ubicada en Moscú, Rusia.

La primera parte del trabajo consiste en implementar una herramienta que nos permita estimar el $Rount Trip Time$ entre nuestro host y distintos hosts, calculando adem\'as el $valor standard$ de  cada salto con respecto a la ruta global.

Luego, procedemos a gr\'aficar y analizar los resultados obtenidos, intentando reconocer enlaces submarinos. Se tendra en cuenta para esto el $valor standard$ calculado por nuestra herramienta y se intentara encontrar un umbral que permita decidir de manera lo suficientemente acertada cuales son los enlaces submarinos.