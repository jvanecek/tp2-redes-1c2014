\section{Introducción}

El objetivo principal de este trabajo práctico es el de analizar el protocolo utilizado a nivel de red. Con ese fin se utilizan diversas herramientas y técnicas. Una de ellas consiste en el $traceroute$ brindado por el sistema operativo y otra la desarrolla en Scapy. 

En primer lugar se seleccionaron tres universidades situadas en otros diferentes continentes. Una de ellas 
es The University of British Columbia, ubicada en la ciudad de Vancouver en Canadá. La segunda es The University of Hong Kong, ubicada en Pokfulam, Hong Kong. Otra de las universidades seleccionadas es Lomonosov Moscow State University, ubicada en Moscú, Rusia. 

La primera parte del trabajo consiste en realizar observaciones sobre el $Round-trip delay time$ de diferentes paquetes cuyos destinos pueden ser algunas de las universidades mencionadas. Se compararán los resultados obtenidos en distintos momentos del día y se generarán diversas conclusiones. También se observará el comportamiento de la señal en cada enlace que atraviesa para poder llegar a destino. 

Por otro lado, la segunda parte consiste en definir los enlaces transatlánticos que va a utilizar un paquete desde el inicio hasta llegar al destino establecido. Para ello se implementará una heurística dada por la cátedra. Luego se propondrán nuevas técnicas para detectar los mismos. 
